\chapter{Numpy and Matplotlib}
\section{Linear Algebra}
This section would only have very basic of linear algebra, adapted from Artin's algebra textbook\cite{Artin}. This is not intended for mathematics course which is why we would keep it to bare minimum

\noindent Let $m$ and $n$ be positive integers. An $m \times n$ matrix is a collection of $mn$ numbers arranged in rectangular array.
\begin{figure}[ht]
\centering
$\begin{bmatrix}
	a_{11} & \cdots & a_{1n} \\
	\vdots &  &  \vdots \\
	a_{m1}& \cdots  &a_{mn}  \\
\end{bmatrix}$
\caption{$m \times n$ matrix.}
\label{fig:mn matrix}
\end{figure}

\noindent For example, $\begin{bmatrix}
	2 & 1 & 0 \\
	1& 3  &5  \\
\end{bmatrix}$ is a $2 \times 3$ matrix.

\noindent Numbers in a matrix are the \textit{matrix entries}. They are usually denoted as $a_{ij}$ where $i$ and $j$ are indices with $1\leq i \leq m $ and$1\leq j \leq n $.

\noindent An $n \times n$ matrix is called \textit{square matrix}. An $1 \times n$ matrix is an $n$-dimensional row vector.

\noindent Let $A=(a_{ij})$ and $B=(b_{ij})$ be two $m \times n$ matrix. Their sum $A+B$ is the $m \times n$ matrix $S = (s_{ij})$ defined by:
\begin{equation}\label{matrix addition}
	s_{ij}=a_{ij}+b_{ij} 
\end{equation}
Thus
\begin{equation}
	\begin{bmatrix}
		2&1  &0  \\
		1& 3 &5  \\
	\end{bmatrix}+\begin{bmatrix}
		1 & 0 & 3 \\
		4 & -3 & 1 \\
	\end{bmatrix}=\begin{bmatrix}
		3 & 1 & 3 \\
		5 & 0 & 6 \\
	\end{bmatrix}
\end{equation}

\noindent Scalar multiplication of an $m \times n$ matrix $A$ by a number $c$ is another $m \times n$ matrix $B=(b_{ij})$, where $(b_{ij}=ca_{ij}$ for all $i,j$. Thus
\begin{equation}
	2\begin{bmatrix}
		2&1  &0  \\
		1& 3 &5  \\
	\end{bmatrix}=\begin{bmatrix}
		4 & 2 & 0 \\
		2 & 6 & 10 \\
	\end{bmatrix}
\end{equation}

\noindent Things start to get nasty when we come to Matrix multiplication. Before we get there, it's imperative to first learn about the most basic form of matrix, vectors.

\noindent Let $A$ be a row vector and $B$ a column vector of the same size, let say $m$. If the entries of $A$ and $B$ are denoted by $a_i$ and $b_i$ respectively, the (dot) product of $AB$ is a $1 \times 1$ matrix or scalar.
\begin{equation}
	\begin{bmatrix}
		a_1& a_2  & \cdots &a_m  \\
	\end{bmatrix}\times
	\begin{bmatrix}
		b_1 \\
		b_2 \\
		\vdots \\
		b_m
	\end{bmatrix}=a_1b_1+a_2b_2+\cdots+a_mb_m
\end{equation}

\noindent Thus

\begin{equation}
	\begin{bmatrix}
		1& 3  & 5  \\
	\end{bmatrix}\times
	\begin{bmatrix}
		1 \\
		-1 \\
		4
	\end{bmatrix}=1-3+20=18
\end{equation}
\noindent The entries of product matrix are computed by multiplying all rows of $A$ by all columns of $B$. If we denote the product matrix $AB$ by $P=(p_{ij})$, then

\begin{equation}
	p_{ij}=a_{i1}b_{1j}+a_{i2}b_{2j}+\cdots+a_{im}b_{mj}
\end{equation}
\noindent For example,

\begin{equation}
	\begin{bmatrix}
		2&1  &0  \\
		1& 3 &5
	\end{bmatrix}
	\begin{bmatrix}
		1 \\
		-1 \\
		4
	\end{bmatrix}=
	\begin{bmatrix}
		1 \\
		18
	\end{bmatrix}
\end{equation}

\noindent Care should be taken that the product of matrix multiplication is non-commutative, $AB\not\equiv BA$ .

\noindent Transpose of a matrix $A$ is a matrix $A^T$ with it's row and column flipped.
\begin{equation}
\begin{bmatrix}
	2&1  &0  \\
	1& 3 &5 
\end{bmatrix}^T=
\begin{bmatrix}
	2& 1 \\
	1& 3 \\
	0&5
\end{bmatrix}
\end{equation}
\noindent Matrix $A$ is called invertible if there is $n\times n$ square matrix $B$ such that $AB=BA=I_n$ where $I_n$ is $n\times n$ identity matrix. For example,
\begin{equation}
	\begin{bmatrix}
		2&1   \\
		5& 3 
	\end{bmatrix}
	\begin{bmatrix}
		3& -1 \\
		-5& 2 
	\end{bmatrix}=
	\begin{bmatrix}
		1& 0 \\
		0& 1 
	\end{bmatrix}
\end{equation}
\noindent Matrices could also be used to solve systems of equation.

\section{Numpy}
Numpy is a python library for handling large, multi-dimensional arrays and matrices, along with a large collection of high-level mathematical functions to operate on these arrays.
\section{Matplotlib}
will be image intensive, set a folder for that
\section{Seaborn}