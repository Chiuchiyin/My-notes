\chapter{Modular Arithmetic}
Reading Materials: \newline
Yan, S.Y. \textit{Number theory for computing.} (Berlin: Springer-Verlag, 2002) 2nd edition. \textbf{Section 1.2 Theory of divisibility pp.21-pp.24 , Section 1.6 Theory of congruences pp.111-119}
\subsection{Congruence}
\noindent Modular arithmetic is a system of arithmetic for integers where the number "wraps around" after reaching a certain value we call \textbf{modulus}.

\vspace{5mm}

\noindent Modular arithmetic is commonly used in number theory, algebra (group theory, ring theory, etc) and also in cryptography. An example in daily life is the clock. The numbers go from 1 to 12, but when you get to "13 o'clock", it actually becomes 1 o'clock again (think of how the 24 hour clock numbering works). So 13 becomes 1, 14 becomes 2, and so on. This can keep going, so when you get to "25 o'clock'', you are actually back round to where 1 o'clock is on the clock face (and also where 13 o'clock was too), this is arithmetic modulo 12.

\vspace{5mm}

\noindent In our clock example, the number go from 1 to 12. In formal mathematics, we usually start from 0. In this case our clock would have 12 replaced with zero. Thus,
\begin{equation}
24\equiv 0 \mod{12}
\end{equation}
\noindent Two integers $a$ and $b$ are said to be \textbf{congruent} modulus $k$ if when they are divided by $k$, they have the same remainder.
\vspace{5mm}
\noindent More examples:
\begin{equation}
\begin{split}
3 &\equiv 5 \mod{2} 
\end{split}
\end{equation}
3 divided by 2 gives 1 with remainder 1, 5 divided by 2 gives 2 with remainder 1.
\begin{equation}
\begin{split}
14 &\equiv 2 \mod{12} 
\end{split}
\end{equation}
14 divided by 12 gives 1 with remainder 2, 2 divided by 12 gives 0 with remainder 2.
\vspace{5mm}
\noindent How about negative numbers? $-17 \mod{12} $ for example:
\begin{equation}
\begin{split}
-17 \mod{12} & \equiv -5\mod{12}\\
 & \equiv 7 \mod{12}
\end{split}
\end{equation}
\section{Operations in Modular Artithmetics}