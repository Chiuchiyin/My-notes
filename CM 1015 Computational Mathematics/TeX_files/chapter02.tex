\chapter{Series and Sequence}
Reading Materials: \newline
Croft, A. and R. Davidson \textit{Foundation maths.} (Harlow: Pearson, 2016) 6th edition. \textbf{Chapter 12 Sequences and series.}
\section{Little Gauss}
\noindent There is this story that is often told in mathematics classes. While the story itself is likely apocryphal, it likely have some pedagogical value. The story goes this way:

\vspace{5mm} %5mm vertical space

\noindent There was once a German school where a boy Carl Friedrich made mischief during mathematics lesson. Instead of corporal punishment that was common in that time, the teacher instead decided to give him mathematics assignment to keep him busy. He was asked to add up the numbers from one to a hundred. Most students would diligently start adding and be busy for a while. The young Carl Friedrich, on the other hand, answered after a few minutes. The teacher was surprised at the request to speak, since he had just kept the boy busy. He was all the more astonished when Carl Friedrich said that he had finished the task and was even able to say the correct result (5050).

\vspace{5mm} %5mm vertical space

\noindent{\large\textbf{How had he solved it?}}

\vspace{5mm} %5mm vertical space

\noindent How he did it so fast? Carl Friedrich discovered discovered the following - unfortunately I do not know what coincidence was behind it. He wrote the numbers down like this:

\begin{center}
	\begin{tabular}{cccccc}
		1   & 2  & 3  & $\mathellipsis$ & 99 & 100 \\
		100 & 99 & 98 & $\mathellipsis$ & 2  & 1  
	\end{tabular}
\end{center}

\noindent This still doesn't look interesting yet. He would then add up the numbers.

\begin{center}
	\begin{tabular}{cccccc}
		1   & 2  & 3  & $\mathellipsis$ & 99 & 100 \\
		100 & 99 & 98 & $\mathellipsis$ & 2  & 1  \\
		101 & 101 & 101 & $\mathellipsis$ & 101 & 101
	\end{tabular}
\end{center}

\noindent Each of them have the sum 101. This looks rather promising. 

\vspace{5mm} %5mm vertical space 

\noindent To sum it up, we write down the numbers from one to one hundred twice, once in increasing order and once in decreasing order, we would then sum them up and we can clearly see that we obtain the sum of $100 \cdot 101$. But we are not finished yet because we counted each numbers twice so we still have to divide the results by two. Then, we would have the sum of numbers from one to a hundred. And that's exactly how Carl Friedrich proceeded. Do we know Carl Friedrich? Hopefully that's the case, because Carl Friedrich was none other than Carl Friedrich Gauss. One of the most important German mathematicians (if not the most important German mathematician).

\vspace{5mm} %5mm vertical space

\noindent{\large\textbf{Let us talk about the formula}}

\vspace{5mm} %5mm vertical space

\noindent Mathematicians love formulas or should I say the general solution of a problem. The sum of the first $n$ of natural numbers follows the formula:
\begin{equation}
	\Sigma = \frac{n\cdot(n+1)}{2}
\end{equation}
\noindent This is not as complicated as it looks. We could for example count the sum of 1 to 150, then we set $n$ equals to 150.
\begin{equation}
	\Sigma = \frac{150\cdot(150+1)}{2}= \frac{150\cdot(151)}{2}= \frac{22650}{2}=11325
\end{equation}
\noindent This formula is today is still affectionately referred as "Der Kleine Gauss", German for "Little Gauss". Anyone studying higher mathematics would have to prove the validity of the formula.