\chapter{Number Bases, Conversion and Operations}

Reading Materials: \newline
Croft, A. and R. Davidson \textit{Foundation maths.} (Harlow: Pearson, 2016) 6th edition. \textbf{Chapter 14 Number Bases}
\section{Number Bases}

\noindent\large\textbf{Decimal System}\newline
\normalsize The numbers that we commonly used are based on 10.\newline
For example:
\begin{equation} \label{eq1}
\begin{split}
253 & = 200 + 50 + 3 \\
& = 2(100) + 5(10) + 3 \\
& = 2(10^2)+ 5(10^1) + 3(10^0)
\end{split}
\end{equation} 

\noindent\large\textbf{Binary System}\newline
\normalsize A binary system uses base 2, it only consist of 2 digits, 0 and 1.\newline
Numbers in base 2 are called binary digits or simply bits.\newline
Consider the binary number $110101_2$. As the base is 2, this means that power of 2 essentially replace powers of 10. Let us convert it to base 10.
\begin{equation} \label{eq2}
\begin{split}
110101_2 & = 1(2^5)+1(2^4)+0(2^3)+1(2^2)+0(2^1)+1(2^0) \\
& = 1(32)+1(16)+0(8)+1(4)+0(2)+1(1) \\
& = 32+16+4+1 \\
& = 53_{10}
\end{split}
\end{equation} 

\noindent\large\textbf{Octal System}\newline
\normalsize Octal numbers use 8 as a base. The eight digits used in the octal system are 0, 1, 2, 3, 4, 5, 6 and 7. Octal numbers use powers of 8, just as decimal numbers use powers of 10 and binary numbers use powers of 2. Example:
\begin{equation} \label{eq3}
\begin{split}
325_8 & = 3(8^2)+2(8^1)+5(8^0) \\
& = 3(64)+2(8)+5(1) \\
& = 192+16+5 \\
& = 213_{10}
\end{split}
\end{equation} 

\noindent\large\textbf{Hexadecimal System}\newline
\normalsize Hexadecimal system use 16 as a base. The digits are 0, 1, 2, 3, 4, 5, 6, 7, 8, 9, A, B, C, D, E and F. Example:
\begin{equation} \label{eq4}
\begin{split}
93\text{A}_{16} & = 9(16^2)+3(16^1)+\text{A}(16^0) \\
& = 9(256)+3(16)+10(1) \\
& = 2304+48+10 \\
& = 2362_{10}
\end{split}
\end{equation} 

\section{Number Conversion}
\subsection{Converting from Decimal to Other Number Base}
\noindent The no-brainer way is to divide the number by the base, the remainder would be the last digit of the new number base. Keep dividing the product until it is smaller than the number base. For example:






\subsection{Conversion with Binary Number}
\noindent Converting binary numbers to Octal or Hexadecimal and vice versa is very straightforward. It can be performed without converting to Decimal first. Let us use number $11100110_2$ as example:
\begin{equation} \label{eq5}
\begin{split}
\underbrace{11}_\text{3}\underbrace{100}_\text{4}\underbrace{110}_\text{6} & = 346_8\\
\underbrace{1110}_\text{E}\underbrace{0110}_\text{6} & = \text{E}6_{16}
\end{split}
\end{equation}
\subsection{Non-integer Number Conversion}
Converting non-integer number might look counterintuitive and intimidating. It is actually rather simple. Let us convert $17.375_{10}$ to binary.
\[\begin{split}
17.375 & = 10 + 7 + 0.3+0.07+0.005 \\
& = 1(10^1) + 7(10^0) + 3(10^{-1}) + 7(10^{-2})+5(10^{-3})
\end{split}\]

\section{Operations with Binary Number}
\section{Common Pitfalls}